\documentclass[11pt, letterpaper]{article}
\usepackage{amsmath, amssymb}
\title{Devoir maison exercice 28 et 29}
\author{Etienne Thomas}
\begin{document}
\maketitle

\section*{Exercice 28:}

1) $f$ est une fonction définie sur $\mathbb{R}_{+}^{*}$ par $f : x \mapsto x + \frac{1}{x}$.
Prouvons que $\forall x \in \mathbb{R}_{+}^{*}, f(x) \geq 2$.

Soit $x \in \mathbb{R}_{+}^{*}$, on a $(x - 1)^2 \geq 0 \Leftrightarrow x^2 - 2x + 1 \geq 0$
\[\Leftrightarrow x^2 + 1 \geq 2x\]
\[\fbox{$\Leftrightarrow x + \frac{1}{x} \geq 2$}\]

Ainsi, $\forall x \in \mathbb{R}_{+}^{*}, f(x) \geq 2$.
\\
2) Montrons que $\forall (a, b) \in (\mathbb{R}_{+}^{*})^2, \frac{a}{b} + \frac{b}{a} \geq 2$.\\
On sait que 
\[ f(\frac{a}{b}) = \frac{a}{b} + \frac{b}{a} \]
et que \[f(\frac{a}{b}) \geq 2\]
Donc \[\fbox{$\frac{a}{b} + \frac{b}{a} \geq 2$}\].

Par conséquent, $\forall (a, b) \in (\mathbb{R}_{+}^{*})^2, \frac{a}{b} + \frac{b}{a} \geq 2$ est verifié.

\section*{Exercice 29:}

1) On cherche a montrer que $\forall x \in \mathbb{R}_{+}, x - \frac{1}{2}x^2 \leq ln(1 + x)$.
Pour cela, on étudie la fonction $f : x \mapsto ln(1 + x) - x + \frac{x^2}{2}$.\\
$ln(x + 1)$ est definie et derivable sur $\mathbb{R}_{+}$, $x - \frac{x^2}{2}$ est un polynome,
donc il est défini et dérivable sur $\mathbb{R}$.
Par somme, $f$ est dérivable sur $\mathbb{R}_{+}$, de dérivée:
\[f'(x) = \frac{1}{x + 1} - 1 + x\]
\[f'(x) = \frac{x^2}{1 + x}\]

$x \in \mathbb{R}_{+} \Rightarrow x^2 \geq 0$ et $ x + 1 > 0$, donc $f'(x) \geq 0$.
Par conséquent $f$ est croissante sur $\mathbb{R}_{+}$.\\
\[\Leftrightarrow (a, b) \in (\mathbb{R}_{+})^2, a \leq b \Rightarrow f(a) \leq f(b)\]
Posons $a = 0$ et $b = x$, on a: \[f(0) = ln(1) - 0 + 0 = 0\]
\[f(0) \leq f(x) \Leftrightarrow 0 \leq f(x)  \]

Par conséquent, $\forall x \in \mathbb{R}_{+}, x - \frac{1}{2}x^2 \leq ln(1 + x)$.

\end{document}

